%% LyX 2.3.6 created this file.  For more info, see http://www.lyx.org/.
%% Do not edit unless you really know what you are doing.
\documentclass[spanish]{article}
\usepackage[T1]{fontenc}
\usepackage{textcomp}
\usepackage{amstext}

\makeatletter
%%%%%%%%%%%%%%%%%%%%%%%%%%%%%% Textclass specific LaTeX commands.
\newenvironment{lyxcode}
	{\par\begin{list}{}{
		\setlength{\rightmargin}{\leftmargin}
		\setlength{\listparindent}{0pt}% needed for AMS classes
		\raggedright
		\setlength{\itemsep}{0pt}
		\setlength{\parsep}{0pt}
		\normalfont\ttfamily}%
	 \item[]}
	{\end{list}}

%%%%%%%%%%%%%%%%%%%%%%%%%%%%%% User specified LaTeX commands.
\usepackage{babel}

\makeatother

\usepackage{babel}
\addto\shorthandsspanish{\spanishdeactivate{~<>}}

\begin{document}
\title{{\Huge{}Practica 1 gr\'{a}maticas y expresiones regulares}}
\author{Guillermo Alejandro Westerhof Rodriguez}
\maketitle

\section{Descripcion de la pr\'{a}ctica:}

En esta practica hay que hacer los dos siguientes ejercicios: (la
potencia de una relacion y buscar con la funcion ``grep'' un archivo
formato .tex

\subsection{Encontrar la potencia de $R^{3}$ de $R=\{(1,1),(1,2),(2,3),(3,4)\}$}
\begin{lyxcode}
Sabiendo~que~$R^{3}=R\text{\textopenbullet}R\text{\textopenbullet}R$,~~$R{{}^2}=R\text{\textopenbullet}R$,~

Entonces:~$R{{}^3}=R{{}^2}\text{\textopenbullet}R=(R\text{\textopenbullet}R)\text{\textopenbullet}R$~~

Siendo~la~operacion~\textopenbullet{}~la~composicion~entre~dos~relaciones~de~conjuntos.
\end{lyxcode}
%
\begin{lyxcode}
Calculamos~primero~$R{{}^2}=\{(1,1),(1,2),(1,3),(2,4)\}$
\end{lyxcode}
%
\begin{lyxcode}
Finalmente~$R^{3}=R{}^{2}\text{\textopenbullet}R=\{(1,1),(1,2),(1,3),(1,4)\}$
\end{lyxcode}

\subsection{Encontrar un archivo .tex con el contenido \textbackslash usepackage\{amsthm,
amsmath\}}
\begin{lyxcode}
~~
\end{lyxcode}

\end{document}
