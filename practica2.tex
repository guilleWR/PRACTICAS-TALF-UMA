%% LyX 2.3.6 created this file.  For more info, see http://www.lyx.org/.
%% Do not edit unless you really know what you are doing.
\documentclass[spanish]{article}
\usepackage[T1]{fontenc}

\makeatletter
%%%%%%%%%%%%%%%%%%%%%%%%%%%%%% Textclass specific LaTeX commands.
\newenvironment{lyxcode}
	{\par\begin{list}{}{
		\setlength{\rightmargin}{\leftmargin}
		\setlength{\listparindent}{0pt}% needed for AMS classes
		\raggedright
		\setlength{\itemsep}{0pt}
		\setlength{\parsep}{0pt}
		\normalfont\ttfamily}%
	 \item[]}
	{\end{list}}

%%%%%%%%%%%%%%%%%%%%%%%%%%%%%% User specified LaTeX commands.
\usepackage{babel}

\makeatother

\usepackage{babel}
\addto\shorthandsspanish{\spanishdeactivate{~<>}}

\begin{document}
\title{{\Huge{}Pr\'{a}ctica 2 Automata en JFLAP}}
\author{Guillermo Alejandro Westerhof Rodr\'{\i}guez}
\maketitle

\section{Descripci\'{o}n de la pr\'{a}ctica:}

En esta pr\'{a}ctica se formar\'{a} un aut\'{o}mata con el alfabeto
$\sum=\{a,b\}$ que solo aceptar\'{a} la cadena formada por 'a'. Lo
haremos en JFLAP en el que podremos tambi\'{e}n hacer diferentes pruebas
y comprobar que realmente solo aceptar\'{a} esa cadena en espec\'{\i}fico.
Al final se implementar\'{a} tambi\'{e}n en Octave, describiendo el
JSON dentro de un entorno 'verbatim' de Latex.

\subsection{Descripci\'{o}n del aut\'{o}mata y visualizaci\'{o}n del diagrama }

\subsubsection*{\textmd{El Aut\'{o}mata finito determinista que haremos ser\'{a}
de la siguiente forma: $M=\{K,\sum,\delta,s,F\}.$ Donde K es el conjunto
total de todos los estados $\{$$q_{0},q_{1},q_{2}\}$, El alfabeto
sigma descrito al principio, la funcion de transici\'{o}n $\delta:K\times\sum\rightarrow K$,
un estado inicial que sera $q_{0}$ y un conjunto de estados finales
$F=\{q_{1}\}$ }}

\section{Resultados obtenidos introduciendo diferentes cadenas}
\begin{lyxcode}

~
\end{lyxcode}

\end{document}
